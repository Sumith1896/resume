
\documentclass{resume2} % Use the custom resume.cls style
\usepackage{graphicx}
\usepackage{enumitem}
\usepackage[left=0.5in,top=0.22in,right=0.5in,bottom=0.4in]{geometry} % Document margins
\linespread{0.85}
\usepackage{setspace}
\usepackage{enumitem}
\usepackage{array}
\usepackage{hyperref}

\begin{document}

\vspace{2.17cm}

\begin{rSection}{Extracurricular Activities}
\begin{itemize}[leftmargin=*]
\item[$\star$] Stood 17th (2nd among teams from IIT-B) in online round of \textbf{ACM ICPC Amritapuri Regionals} (2013)
\item[$\star$] Coordinator at \textbf{Mood Indigo} and \textbf{Techfest}, which are respectively the largest cultural and technology college festivals in Asia 
\item[$\star$] Invited to \textbf{Vijyoshi} camp (2011) at \textbf{IISc Bangalore} (funded by Dept. of Science and Technology, India)
\item[$\star$] Attended the \textbf{summer program} for KVPY fellows at \textbf{IISER Mohali} (2011)
\item[$\star$] Served as \textbf{Vice-President} of Axiom (school mathematics society) and was responsible for the execution of Axiom 2011 (second largest inter-school mathematics symposium in Delhi) and won several inter-school mathematics competitions
\item[$\star$] Made a \textbf{line-following} bot for a competition organized by Students Technical Activities Body
\item[$\star$] \textbf{2nd Position} in team \textbf{Math and Physics quiz} conducted by Math and Physics Club, IIT Bombay
\item[$\star$] Represented Hostel-2 in inter hostel coding general championship
\item[$\star$] Active participant in algorithmic coding contests on online judges (\href{http://spoj.com/users/pratyaksh}{SPOJ}, Codechef, \href{codeforces.com/profile/pratyaksh}{Codeforces})
\item[$\star$] Served one year under the \textbf{National Service Scheme} (NSS) (2012-13)
\end{itemize}
\end{rSection}

\begin{rSection}{Scholastic Achievements}
\begin{itemize}[leftmargin=*]
\item[$\star$] \textbf{All India Rank 33} in \textbf{IIT Joint Entrance Examination} (2012) out of over 500,000 candidates
\item[$\star$] \textbf{All India Rank 29} in \textbf{AIEEE} (2012) out of over 1,200,000 candidates
\item[$\star$] Qualified for \textbf{Indian National Physics (INPhO), Chemistry (INChO) and Astronomy (INAO) Olympiads} (2012) (\textbf{top 300} students in India in each subject) and awarded certificates of merit for being in \textbf{top 1\%} students
\item[$\star$] Qualified for the \textbf{Indian National Mathematics Olympiad} after clearing RMO in Delhi (2011)
\item[$\star$] Awarded the \textbf{Kishore Vaigyanik Protsahan Yojana} (KVPY) scholarship by Department of Science and Technology of the Government of India, aimed at encouraging students to take up research careers
\item[$\star$] \textbf{Honourable Mention} at \textbf{ACM-ICPC} Amritapuri Regionals, Onsite Rank-24, Online Rank-17
\item[$\star$] Admitted to the \textbf{Education Program for Gifted Youth} (EPGY) Summer
Institutes High School Program at \textbf{Stanford University} (2010)
\item[$\star$] Awarded \textbf{Gold medal} by DPS Society for maintaining excellent academic record for seven consecutive years
\end{itemize}
\end{rSection}

%-------------

\begin{rSection}{Projects and Internships}

\begin{rSubsection}{\href{http://www.ics.uci.edu/~pratyas/report2014.pdf}{Inference in Probabilistic Graphical Models}}{May 2014 - Jul 2014}{Guide: Prof. Rina Dechter (University of California, Irvine)
}{}
\item[$\star$] Performed empirical analysis of \textbf{anytime weighted best first strategies on AND/OR search spaces} for approximate inference 
\item[$\star$] Compared the effects of Mini-bucket Elimination, Join-Graph Linear Programming(JGLP) and MBE with moment matching (MBE-MM) heuristics on search performance
\item[$\star$] Participated in \textbf{UAI 2014 - Probabilistic Inference Competition}
\item[$\star$] Currently studying \textbf{weighted schemes with dynamic weights}, which may lead to a smaller search space
\item[$\star$] \emph{Url:} \url{http://www.ics.uci.edu/~pratyas/report2014.pdf}
\end{rSubsection}

\clearpage

\begin{rSubsection}{Sentiment analysis of song lyrics}{Aug 2014 - Present}{Guide: Prof. Pushpak Bhattacharyya, IIT Bombay}{}
\item[$\star$] Developing a system to analyse song lyrics and output the mood/emotion/sentiment associated with the song
\item[$\star$] Classifying a song solely based on its lyrics is challenging. All current algorithms have poor accuracy
\end{rSubsection}


\begin{rSubsection}{\href{https://github.com/pratyakshs/Che.ss}{Chess}}{Jan 2013 - Apr 2013}{Guide: Prof. Amitabha Sanyal, IIT Bombay}{}
\item[$\star$] Developed a chess engine in the \textbf{functional programming} paradigm using PLT Scheme
\item[$\star$] The \textbf{minimax algorithm} with \textbf{alpha-beta pruning} formed the basis of the AI and also investigated other chess algorithms like \textbf{Negascout} and \textbf{MTD-f} \item[$\star$] The heuristics involved use of Piece-Square tables, analyzing pawn structure and various other chess tactics.
\item[$\star$] Used the \textbf{XBoard} GUI, and made a parser for the \textbf{Chess Engine Communication Protocol}
\item[$\star$] \emph{Url:} \url{https://github.com/pratyakshs/Che.ss}
\end{rSubsection}





\begin{rSubsection}{\href{https://github.com/pratyakshs/Box2D-SteamEngine}{Simulation in Box2D}}{Mar 2014 - Apr 2014}{Guide: Prof. Parag Chaudhuri, IIT Bombay}{}
\item[$\star$] Developed a complex simulation of a locomotive in \textbf{Box2D Physics Engine} 
\item[$\star$] \emph{Url:} \url{https://github.com/pratyakshs/Box2D-SteamEngine}
\end{rSubsection}


%\begin{rSubsection}{Android PC controller}{May 2013 - Jun 2013}{Institute Technical Summer Project}{}
%\item[$\star$] Developed, in a team of four, an \textbf{Android application} to control a PC, using a mobile device
%\item[$\star$] Studied Android \textbf{bluetooth socket programming} and used Java's Robot class
%\end{rSubsection}

%\begin{rSubsection}{Pac-man}{Aug 2012 - Nov 2012}{Guide: Prof. Abhiram Ranade (CSE Dept., IIT Bombay)
%}{}
%\item[$\star$] Recreated the classic game of pac-man using \textbf{Object Oriented Programming} in C++ using the EzWindows graphics library.
%\item[$\star$] Programmed the deterministic ghost behaviour as in the standard game of pac-man
%\end{rSubsection}

\begin{rSubsection}{News clustering hack}{Aug 2013}{Yahoo! HackU}{}
\item[$\star$] Prototype for \textbf{clustering} similar \textbf{news content} in Yahoo! News results using several Yahoo! APIs like BOSS, YQL and YUI
\end{rSubsection}

\begin{rSubsection}{\href{https://github.com/pratyakshs/JTalk}{JTalk - P2P chat program}}{Mar 2014}{}{}
\item[$\star$] Developed a text based LAN messenger in Java using multiple threads and the Swing library
\item[$\star$] \emph{Url:} \url{https://github.com/pratyakshs/JTalk}
\end{rSubsection}

\begin{rSubsection}{Pac-man}{Aug 2012 - Nov 2012}{Guide: Prof. Abhiram Ranade (CSE Dept., IIT Bombay)
}{}
\item[$\star$] Recreated the classic game of pac-man using in C++ using the \textbf{EzWindows graphics library}.
\item[$\star$] Programmed the deterministic ghost behaviour as in the standard game of pac-man
\end{rSubsection}

\begin{rSubsection}{Analytics developer}{Jul 2013 - Sep 2013}{\href{http://coursewave.org/}{Coursewave.org}, an online courseware startup}{}
\item[$\star$] Developed live data visualizations for students and teachers using various web APIs
\item[$\star$] Integrated the analytics with online test taking interface
\end{rSubsection}

\end{rSection}

%--------------

\begin{rSection}{Technical Skills}

\begin{tabular}{ @{} >{\bfseries}l @{\hspace{6ex}} l }
Computer Languages & C/C++, Python, PLT Scheme, Prolog, Java, Bash, VHDL, MIPS\\
Web \& APIs & HTML, CSS, Javascript, PHP, MySQL, AJAX, JSON \\
Tools \& Libraries &  Sage, \LaTeXe , Scilab, PSPICE, Mathematica, Numpy, Boost C++
\end{tabular}
\\
\end{rSection}

%---------------



\begin{rSection}{Relevant Courses Undertaken}
\begin{rSubsection}{Computer Science:}{}{}{}
\item[$\star$] Computer Architecture*, Computer Networks*, Database Systems*, Natural Language Processing*, Network Security and Cryptography*, Design and Analysis of Algorithms, Introduction to Machine Learning, Data Structures and Algorithms, Abstractions and
Paradigms in Programming, Discrete Structures, Graph Theory, Number Theory and Cryptography, Automata Theory and Logic, Logic
Design, Software Systems Laboratory

\end{rSubsection}

\begin{rSubsection}{Statistics and Mathematics:}{}{}{}
\item[$\star$] Statistical Inference*, Game Theory*, Calculus, Linear Algebra, Differential Equations, Data Analysis and Interpretation, Probability Theory, Combinatorics, Numerical Analysis, Introduction to Derivative Pricing
\end{rSubsection}

\begin{rSubsection}{Other courses:}{}{}{}
\item[$\star$]  Electrical and Electronic Circuits, Modern Physics, Economics, Experimental and Measurement Laboratory
\end{rSubsection}

\textbf{Note:} *-will be completed by Dec 2014.
\end{rSection}

\end{document}

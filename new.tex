
\documentclass{resume2} % Use the custom resume.cls style
\usepackage{graphicx}

\usepackage[left=0.5in,top=0.4in,right=0.5in,bottom=0.4in]{geometry} % Document margins
\linespread{0.85}
\usepackage{setspace}
\usepackage{enumitem}

\name{PRATYAKSH SHARMA}
\begin{document}

\begin{rSection}{Education}
Second year undergraduate, Indian Institute of Technology, Bombay \\B.Tech. with Honors in Computer Science and Engineering \\ Minor in Applied Statistics and Informatics. \\ CPI: 9.16/10.0
\end{rSection}

\begin{rSection}{Projects}

\begin{rSubsection}{Chess}{Jan 2013 - Apr 2013}{Guide: Prof. Amitabha Sanyal (HOD, CSE Dept., IIT Bombay)
}{}
\item Developed a chess engine in the \textbf{functional programming} paradigm using several higher order functions and abstractions
\item The \textbf{minimax algorithm} with \textbf{alpha-beta pruning} formed the basis of the AI and also investigated other chess algorithms like \textbf{Negascout} and \textbf{MTD-f} \item The heuristics involved use of Piece-Square tables, analyzing pawn structure and various other chess tactics. 
\item Used the \textbf{XBoard} GUI, and made a parser for the \textbf{Chess Engine Communication Protocol}.
\end{rSubsection}


\begin{rSubsection}{Android PC controller}{May 2013 - Jun 2013}{Institute Technical Summer Project}{}
\item Developed, in a team of four, an \textbf{Android application} to control a PC, using a mobile device
\item Studied Android \textbf{bluetooth socket programming} and used Java's Robot class

\end{rSubsection}

\begin{rSubsection}{Pac-man}{Aug 2012 - Nov 2012}{Guide: Prof. Abhiram Ranade (CSE Dept., IIT Bombay)
}{}
\item Recreated the classic game of pac-man using \textbf{Object Oriented Programming} in C++ using the EzWindows graphics library.
\item Programmed the deterministic ghost behaviour as in the standard game of pac-man

\end{rSubsection}


\begin{rSubsection}{A-Pathshala}{Aug 2013 - Present}{Guide: Prof. Deepak B. Phatak (Chair Professor, CSE Dept., IIT Bombay)}{}
\item Developing, in a team of four, a \textbf{learning platform} based on the Aakash Android tablet (\textbf{Ministry of Human Resource Development}, Government of India) using \textbf{Apache Nutch} webcrawler
\end{rSubsection}

\begin{rSubsection}{News clustering hack}{Aug 2013}{Yahoo! HackU}{}
\item Prototype for \textbf{clustering} similar \textbf{news content} in Yahoo! News results using several Yahoo! APIs like BOSS, YQL and YUI
\end{rSubsection}

\begin{rSubsection}{Analytics developer}{Jul 2013 - Present}{Coursewave.org, an online courseware startup}{}
\item Responsible for developing live data visualizations for students and teachers using various web APIs
\item Integrated the analytics with online test taking interface
\end{rSubsection}

\end{rSection}

%--------------

\begin{rSection}{Areas of Interest}
Algorithms, Functional Programming, Combinatorics
\end{rSection}
%--------------

\begin{rSection}{Technical Skills}

\begin{tabular}{ @{} >{\bfseries}l @{\hspace{6ex}} l }
Computer Languages & C/C++, PLT Scheme, Prolog, Java, Python, Arduino \\
Web \& APIs & HTML, CSS, Javascript, PHP, MySQL, AJAX, JSON \\
Tools & Apache Nutch, Apache Solr, \LaTeXe , Scilab, PSPICE, Mathematica
\end{tabular}
\\\\
\end{rSection}

%---------------

\begin{rSection}{Scholastic Achievements}
\begin{itemize}[leftmargin=*]
\item \textbf{All India Rank 33} in \textbf{IIT Joint Entrance Examination} (2012) out of over 500,000 candidates
\item \textbf{All India Rank 29} in \textbf{AIEEE} (2012) out of over 1,200,000 candidates
\item Qualified for \textbf{Indian National Physics (INPhO), Chemistry (INChO) and Astronomy (INAO) Olympiads} (2012) (\textbf{top 300} students in India in each subject) and awarded certificates of merit for being in \textbf{top 1\%} students
\item Qualified for the \textbf{Indian National Mathematics Olympiad} after clearing RMO in Delhi (2011)
\item Awarded the \textbf{Kishore Vaigyanik Protsahan Yojana} (KVPY) scholarship by Department of Science and Technology of the Government of India, aimed at encouraging students to take up research careers
\item \textbf{Honourable Mention} at \textbf{ACM-ICPC} Amritapuri Regionals, Onsite Rank-24, Online Rank-17
\item Admitted to the \textbf{Education Program for Gifted Youth} (EPGY) Summer
Institutes High School Program at \textbf{Stanford University} (2010)
\item Awarded \textbf{Gold medal} by DPS Society for maintaining excellent academic record for seven consecutive years
\end{itemize}
\end{rSection}



\begin{rSection}{Extracurricular Activities}
\begin{itemize}[leftmargin=*]
\item Stood 17th (2nd among teams from IIT-Bombay) in online round of \textbf{ACM ICPC Amritapuri Regionals} (2013)
\item Coordinator at \textbf{Mood Indigo} and \textbf{Techfest}, which are respectively the largest cultural and technology college festivals in Asia 
\item Attended \textbf{Vijyoshi} camp (2011) at \textbf{IISc Bangalore} (funded by Dept. of Science and Technology, Govt. of India)
\item Attended the \textbf{summer program} for KVPY fellows at \textbf{IISER Mohali} (2011)
\item Served as \textbf{Vice-President} of Axiom (school mathematics society) and was responsible for the execution of Axiom 2011 (second largest inter-school mathematics symposium in Delhi) and won several inter-school mathematics competitions
\item Made a \textbf{line-following} bot for a competition organized by Students Technical Activities Body (STAB, IIT Bombay)
\item Active participant in algorithmic coding contests on online judges (SPOJ, Codechef, Codeforces)
\item Served one year under the \textbf{National Service Scheme} (NSS) (2012-13)
\item \textbf{Adventure sports} (paragliding, skydiving, kayaking and hiking) enthusiast
\end{itemize}
\end{rSection}

\begin{rSection}{Relevant Courses Undertaken}
\begin{rSubsection}{Computer Science:}{}{}{}
\item Design and Analysis of Algorithms*, Introduction to Machine Learning*, Data Structures and Algorithms, Abstractions and
Paradigms in Programming, Discrete Structures, Graph Theory*, Number Theory and Cryptography*, Automata Theory and Logic*, Logic
Design*, Software Systems Laboratory*, Computer Programming and Utilization

\end{rSubsection}

\begin{rSubsection}{Statistics and Mathematics:}{}{}{}
\item Calculus, Linear Algebra, Differential Equations, Data Analysis and Interpretation, Introduction to Probability Theory, Combinatorics, Numerical Analysis*, Introduction to Derivative Pricing*
\end{rSubsection}

\begin{rSubsection}{Other courses:}{}{}{}
\item Introduction to Electrical and Electronic Circuits, Modern Physics, Economics, Experimental and Measurement Laboratory
\end{rSubsection}

\begin{rSubsection}{Online Courses (coursera.org)}{}{}{}
\item Algorithms I and II (Princeton University), Game Theory* (Stanford University), Introduction to Logic* (Stanford University), Machine Learning* (Stanford University)
\end{rSubsection}
\textbf{Note:} *-will be completed by April 2014.
\end{rSection}

\end{document}
